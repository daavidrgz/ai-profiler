%%%%%%%%%%%%%%%%%%%%%%%%%%%%%%%%%%%%%%%%%%%%%%%%%%%%%%%%%%%%%%%%%%%%%%%%%%%%%%%%

\pagestyle{empty}
\begin{abstract}
  Con el objetivo de proporcionar una herramienta útil basada en el perfilado automático de autores, a lo largo de este trabajo
  se profundizará más en este campo en el contexto de las redes sociales, analizando cuáles son los mejores algoritmos,
  las técnicas más utilizadas y las características más comúnes que se suelen emplear.
  Asimismo, se desarrollará, bajo la metodología Scrum, una aplicación web en la que se permita a los usuarios perfilar los autores de una
  colección de textos propia —haciendo uso de algoritmos seleccionados en la fase previa de investigación— y mostrando los resultado del perfilado en un \textit{dashboard} intuitivo y accesible.
  Finalmente, para mostrar un caso de uso real de la aplicación, se empleará la colección de referencia sobre el movimiento \textit{Black Lives Matter} (\#BLM)
  puesta a disposición para este TFG por los tutores del mismo, con el objetivo de analizar el perfil de los usuarios que participaron en los debates en las redes sociales sobre este movimiento
  de gran relevancia social.

  \vspace*{25pt}
  \begin{segundoresumo}
    To provide a useful tool based on automatic author profiling, throughout this work, we will research deeper into this field in the context of social networks, analyzing the best algorithms,
    the most commonly used techniques and the typical features that are often employed. Furthermore, under the Scrum methodology, we will
    develop a web application that allows users
    to profile authors of their own text datasets —employing algorithms selected in the previous research phase— and displaying the profiling results on an intuitive and accessible dashboard.
    Finally, to illustrate a real use case of the application, the reference collection related to the "Black Lives Matter" movement (\#BLM) will be utilized. This collection has been made available for
    this Bachelor's Thesis by its supervisors, with the aim of analyzing the profiles of users who participated in social media discussions about this socially significant movement.
  \end{segundoresumo}
  \vspace*{25pt}
  \newpage
\begin{multicols}{2}
\begin{description}
\item [\palabraschaveprincipal:] \mbox{} \\[-20pt]
  \begin{itemize}
    \item \textit{Perfilado de autor}
    \item \textit{Redes sociales}
    \item \textit{Procesado de lenguaje natural}
    \item \textit{Aplicación web}
    \item \textit{Aprendizaje automático}
  \end{itemize}
\end{description}

\begin{description}
\item [\palabraschavesecundaria:] \mbox{} \\[-20pt]
  \begin{itemize}
    \item \textit{Author profiling}
    \item \textit{Social networks}
    \item \textit{Natural language processing}
    \item \textit{Web application}
    \item \textit{Machine learning}
  \end{itemize}
\end{description}
\end{multicols}

\end{abstract}
\pagestyle{fancy}

%%%%%%%%%%%%%%%%%%%%%%%%%%%%%%%%%%%%%%%%%%%%%%%%%%%%%%%%%%%%%%%%%%%%%%%%%%%%%%%%
