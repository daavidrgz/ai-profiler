\chapter{Metodología}
\label{chap:metodologia}

Para el desarrollo de este proyecto se ha utilizado una metodología ágil, concretamente Scrum \cite{scrum}.
Esta metodología se basa en la realización de iteraciones cortas, llamadas \textit{sprints}, en las que se desarrolla una parte del proyecto
denominada incrementeo, es decir, una versión entregrable del producto que contiene
nuevas funcionalidades o mejoras de las ya existentes. La decisión de optar por una metodología de tipo ágil frente a una tradicional
está condicionada por el tiempo de desarrollo corto, de apenas cuatro meses, y por los cambios que se podían producir en los requisitos.
Además, todo el equipo se sentía más cómodo con esta metodología dado que, a mayores de tener experiencia anterior en su uso, se realizaron
ligeras modificaciones a la metodología comentadas en la Sección \ref{sec:metodologia_eventos} que facilitaron en gran medida el desarrollo del
proyecto.

\section{Roles}
\label{sec:metodologia_roles}

En Scrum existen tres roles principales:

\begin{itemize}
		\item \textbf{Product Owner}: Como su nombre indica, es el propietario del producto, por lo que es el encargado de identificar
		las necesidades de los clientes así como de definir y gestionar el \textit{Product Backlog}, esto es, la lista de requisitos del producto ordenados por prioridad.
		Este rol es desempeñado por el autor de este documento.
		\item \textbf{Scrum Master}: Su rol principal es el de asegurar que el equipo de desarrollo sigue la metodología Scrum y no se producen
		desajustes en el transcurso del proyecto. Los encargados de asumir este rol fueron Patricia Martín Rodilla y David Otero Freijeiro, los
		tutores de este trabajo.
		\item \textbf{Equipo de desarrollo}: Formado por un equipo normalmente de entre tres y nueve personas, es el encargado de desarrollar el producto
		en cada \textit{sprint} cumpliendo los requisitos establecidos en el \textit{Product Backlog}. En este caso, el equipo de desarrollo está
		formado únicamente por el autor de este documento. Esto tiene una implicación directa en el transcurso del proyecto, ya que,
		al solo contar con un desarrollador, cualquier imposibilidad de este para realizar su trabajo conlleva inevitablemente a una parada en todo el desarrollo,
		aumentando así de forma considerable el riesgo del proyecto.
\end{itemize}

\section{Eventos}
\label{sec:metodologia_eventos}

Para agilizar el desarrollo y no interferir en el trabajo diario del equipo por las obligaciones
externas de cada uno, tanto las reuniones de planificación como las de revisión y retrospectiva se realizaron
el mismo día que se iniciaba cada \textit{sprint}, esto es, aproximadamente cada tres semanas, lo que se ajusta a la duración recomendada
por la metodología Scrum. Estas reuniones son, por lo tanto, de una mayor extensión que las \textit{dailies} clásicas y son también empleadas
para realizar una revisión del incremento desarrollado en el úlitmo \textit{sprint}. Asimismo, para facilitar el hecho de que todos los miembros
del equipo pudiesen asistir a dichas reuniones, se realizaban de forma telemática haciendo uso de Microsft Teams, como se comentó en la Sección \ref{sec:herramientas_soporte}.

\bigskip
En lo que respecta a los \textit{sprints}, como se detallará en la Sección \ref{sec:planificacion_temporal}, comentar que los dos primeros
no atienden específicamente al desenvolvimiento de ninguna historia de usuario, sino que se emplean para la configuración del entorno de desarrollo
y para la experimentación con los algoritmos de perfilado encontrados. Por lo tanto, el desarrollo de las historias de usuario se inicia
a partir del tercer \textit{sprint}, donde ya se sigue de forma habitual el desarrollo clásico bajo la metodología Scrum.
