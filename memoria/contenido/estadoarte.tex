\chapter{Estado del arte del perfilado de autor}
\label{chap:estadoarte}

Durante los inicios del perfilado automático de autor, los algoritmos se centraban en la tarea de la clasificación por género.
En esta línea, trabajos como (Koppel et al., 2002)\cite{koppel2002automatically} se desmarcaban de la tendencia de la época,
la cual se basaba en la clasificación de textos en base a su contenido, para centrarse en la clasificación de textos \textbf{en base a su estilo}. En este caso, se centraban en la obtención del género del autor mediante el análisis
de 920 documentos de carácter formal escritos en inglés con una media de alrededor de 34.300 palabras cada uno, obteniendo una precisión en la clasificación de
aproximadamente el 77\%.

\bigskip
Así, la demostración de la existencia de rasgos diferenciadores en la escritura que permitían perfilar ciertos ascpectos del individuo, especialmente del género, 
supuso un gran avance en el campo del perfilado de autor
y dió pié a la realización de trabajos como (Argamon et al., 2003)\cite{argamon2003gender}, (Corney et.al, 2002)\cite{corney2002gender} o (Otterbacher et al., 2010)\cite{otterbacher2010inferring}, 
así como también permitió el inicio de una clasificación más compleja en base a otras características como la edad, la orientación sexual o la personalidad.

\bigskip
Más tarde, en el año 2011 se celebraría el primer evento organizado por el \textit{PAN} (\textit{Plagiarism Analysis, Authorship Identification, and Near-Duplicate Detection}) \cite{pan},
un foro de investigación que organiza eventos científicos y tareas anuales relacionadas con el análisis forense de textos digitales
y rasgos estilométricos. La primera de estas tareas centrada en el perfilado de autor se celebraría en el año 2013 (Rangel et al., 2013)\cite{rangel2013overview},
en la que se pedía a los participantes que obtuvieran, a partir de una serie de \textit{tweets}, la edad y el género de su autor. El ganador de este concurso obtuvo una
precisión del 60\% en la clasificación de género y del 67\% en la clasificación de edad, haciendo uso, la mayor parte de los participantes, de técnicas de aprendizaje
supervisado como los Árboles de Decisión (en inglés \textit{Decission Trees}) o las Máquinas de Soporte Vectorial (en inglés \textit{Support Vector Machines}) e inluyendo
en sus modelos características basadas en el TF-IDF, n-gramas, etiquetas POS o características como el número de emoticonos o la frecuencia de signos de puntuación.
En los siguientes años se celebrarían nuevas ediciones de esta tarea (Rangel et al., 2014\cite{rangel2014overview}, Rangel et al., 2015\cite{rangel2015overview},
Rangel et al., 2016\cite{rangel2016overview}...), añadiendo nuevas sub-tareas como el reconocimiento de rasgos personales, la ocupación o los dialectos del lenguaje,
así como también alcanzando mejores resultados en la clasificación.

\section{Algoritmos supervisados}

TODO
% Todos estos algoritmos se basan en lo que en recuperación de información se conoce como \textit{TF-IDF} (del inglés \textit{Term Frequency-Inverse Document Frequency}), 
% que es una medida numérica que expresa cuán relevante es una palabra para un documento en una colección. 
% La importancia aumenta proporcionalmente al número de veces que una palabra aparece en el documento, pero se compensa con la frecuencia 
% de la palabra en la colección de documentos, lo que permite manejar el hecho de que algunas palabras son generalmente 
% más comunes que otras.
