\chapter{Implementación}
\label{chap:implementacion}

\lettrine{A} lo largo de este capítulo, se expondrán las decisiones de implementación más relevantes tomadas durante el desarrollo de la aplicación.
y se detallará la estructura de directorios que se ha seguido para organizar el código fuente del proyecto. Finalmente, se hablará
de la importancia del código abierto y de como contribuye a la innovación en el campo de la informática.

\section{Adaptación de los algoritmos}
\label{sec:adaptacion_algoritmos}

{
	\color{blue}
	Dado que el código fuente de los algoritmos de perfilado seleccionados estaba disponible para su uso, fue necesario adaptarlo
	para ofrecer una interfaz comúna a través de la cual se pudiese comunicar el resto de la aplicación.
	Para ello se barajaron en un inicio dos aproximaciones diferentes. La primera de ellas pasaba
	por implementar APIs sencillas en cada uno de los algoritmos, es decir, crear microservicios que se encargasen de recibir
	las peticiones y devolver los resultados al \textit{backend}. La segunda aproximación consistía en adaptar el propio código
	fuente para que cada algoritmo formase parte de una clase que implemente la interfaz mencionada en el diagrama de la Figura \ref{fig:clases_backend}
	(\textit{ProfilingAlgorithm}). Finalmente, se optó por esta segunda opción ya que, a pesar de ser más compleja, permitía
	una mayor integración con el resto de la aplicación y un mejor rendimiento debido a la eliminación de la latencia de red. Además,
	podíamos aprovechar el hecho de que los algoritmos estaban programados en Python para utilizarlos directamente
	en el \textit{backend} sin hacer uso de llamadas interlenguaje.

	\bigskip
	Sin embargo, esta decisión implicó comprender a muy bajo nivel como estaban implementados dichos algoritmos para realizar las modificaciones necesarias
	e implementar nuevas funciones. Esta adaptación del código fue, en el caso del algoritmo de Grivas et al. \cite{grivas2015author},
	bastante compleja, ya que conllevó sustituír librerías obsoletas, reimplementar funciones y resolver los cambios disruptivos (\textit{breaking changes} en inglés) que suponía
	dar el salto desde la versión 2.7 de Python a la 3.10.
}

\section{Estructura del proyecto}
\label{sec:estructura_proyecto}

En relación a la estructura del proyecto, se ha optado por agrupar el código fuente de todo el sistema
bajo un mismo directorio, como se muestra en el diagrama de la Figura \ref{fig:estructura_proyecto},
dado que facilita en gran medida su portabilidad y su mantenimiento al tratarse de un proyecto relativamente
pequeño.

\bigskip
En primer lugar, situados en la raíz del proyecto, podemos diferenciar las carpetas principales que lo conforman, esto es,
\textit{backend}, \textit{fronted} y \textit{datasets}, así como dos archivos:

\begin{itemize}
	\item \textbf{\textit{README.md}}: Contiene un resumen en formato Markdown \cite{markdown} de lo que es la aplicación, de sus características y de la forma de ejecutarla.
	\item \textbf{\textit{docker-compose.yml}}: Es el archivo que almacena las configuraciones de los contenedores de Docker que se deben levantar para que el sistema funcione adecuadamente.
\end{itemize}

\bigskip
Continuando con el directorio \textit{backend}, distinguimos varias carpetas y archivos:

\begin{itemize}
	\item \textbf{\textit{requirements.txt}}: Contiene todas las dependencias de Python que necesita el \textit{backend} para funcionar.
	\item \textbf{\textit{Dockerfile}}: Es el archivo en el que se definen los pasos necesarios, es decir, los comandos a ejecutar en una
	      máquina recién instalada, para construir la imagen de Docker del \textit{backend}.
	\item \textbf{\textit{docker-compose.yml}}: A diferencia del situado en la raíz, este archivo contiene las configuraciones de los contenedores necesarios
	      para el correcto funcionamiento del entorno de desarrollo en el \textit{backend}.
	\item \textbf{\textit{/venv}}: Dado que el proyecto está creado haciendo uso de un entorno virtual para aislar las dependencias de Python, este directorio
	      contiene los archivos para su puesta en marcha.
	\item \textbf{\textit{/src}}: Aquí se almacena todo el código fuente del \textit{backend}, donde podemos encontrar las siguientes carpetas principales:
	      \begin{itemize}
		      \item \textbf{\textit{/application}}: Contiene los controladores y, en general, lo que se encarga de gestionar las comunicaciones exteriores con el sistema.
		      \item \textbf{\textit{/domain}}: Contiene las clases que conforman la capa de dominio, es decir, las entidades, los repositorios, los servicios, los algoritmos
		            y los conversores.
		      \item \textbf{\textit{/infraestructure}}: Contiene los repositorios de tecnologías concretas que implementan las interfaces definidas en la capa de dominio.
		      \item \textbf{\textit{/utils}}: Contiene clases de utilidad requeridas en diversas partes del sistema.
		      \item \textbf{\textit{main.py}}: Se corresponde con el punto de entrada de la aplicación.
	      \end{itemize}
\end{itemize}

\bigskip
Por otro lado, la estructura que sigue el \textit{frontend}, muy ligada al \textit{framework} de dearrollo NextJS, es la siguiente:

\begin{itemize}
	\item \textbf{\textit{package.json}}: En este archivo se almacenan todas las dependencias de NodeJS que requiere el \textit{frontend}.
	\item \textbf{\textit{next.config.js}}: La importancia de este archivo, más allá de contener las configuraciones básicas de NextJS, reside en la posibilidad
	      de definir un \textit{reverse proxy} capaz de redirigir las peticiones al \textit{backend} a la máquina local durante el desarrollo o a la máquina remota,
	      en este caso otro contenedor de Docker, en producción.
	\item \textbf{\textit{Dockerfile}}: Archivo análogo al presente en el directorio \textit{backend}, necesario para la construcción de la imagen del \textit{frontend}.
	\item \textbf{\textit{/src}}: Dentro de esta carpeta, nos encotramos la estructura típica de un proyecto en NextJS:
	      \begin{itemize}
		      \item \textbf{\textit{/components}}: Contiene los componentes que conforman las páginas de la aplicación.
		      \item \textbf{\textit{/model}}: Contiene las clases que representan las entidades del sistema en el \textit{frontend}.
		      \item \textbf{\textit{/pages}}: Contiene las páginas navegables de la aplicación, estructuradas en directorios según su ruta.
		      \item \textbf{\textit{/services}}: Contiene los servicios encargados de abstraer las peticiones que se realizan al \textit{backend}.
		      \item \textbf{\textit{/styles}}: Contiene las hojas de estilos globales de la aplicación.
		      \item \textbf{\textit{/utils}}: Contiene funciones de diversa utilidad en el proyecto.
	      \end{itemize}
\end{itemize}

\bigskip
Finalmente, el directorio \textit{datasets} se corresponde con el lugar donde se agrupan todas las colecciones que se han utilizado
para el entrenamiento y la validación de los modelos. Entre ellos se incluye una gran parte de \textit{datasets} ofrecidos por PAN \cite{pan} en las competiciones
de perfilado de autor que organiza, además de otras como la del movimiento \#BLM ofrecida en este trabajo y de la que se profundizará más en la Sección \ref{sec:casouso_dataset}.

\bigskip
\begin{figure}[H]
	\centering
	\begin{subfigure}[T]{0.5\textwidth}
		\begin{forest}
			for tree={
			font=\ttfamily\tiny,
			grow'=0,
			child anchor=west,
			parent anchor=south,
			inner xsep=10pt,
			anchor=west,
			calign=first,
			edge path={
					\noexpand\path [draw, \forestoption{edge}]
					(!u.south west) +(7.5pt,0) |- node[fill,inner sep=1.25pt] {} (.child anchor)\forestoption{edge label};
				},
			before typesetting nodes={
					if n=1
						{insert before={[,phantom]}}
						{}
				},
			fit=band,
			s sep=8pt,
			before computing xy={l=15pt},
			}
			[root, fill=gray!20
			[README.md, fill=gray!20]
			[docker-compose.yml, fill=gray!20]
			[/backend, for tree={fill=blue!20}
			[/src
			[/application]
			[/domain
			[/algorithms]
			[/converters]
			[/entities]
			[/repositories]
			[/services]
			]
			[/infraestructure]
			[/utils]
			[/main.py]
			]
			[/venv]
			[docker-compose.yml]
			[Dockerfile]
			[requirements.txt]
			]
			]
		\end{forest}
	\end{subfigure}%
	\begin{subfigure}[T]{0.5\textwidth}
		\begin{forest}
			for tree={
			font=\ttfamily\tiny,
			grow'=0,
			child anchor=west,
			parent anchor=south,
			inner xsep=10pt,
			anchor=west,
			calign=first,
			edge path={
					\noexpand\path [draw, \forestoption{edge}]
					(!u.south west) +(7.5pt,0) |- node[fill,inner sep=1.25pt] {} (.child anchor)\forestoption{edge label};
				},
			before typesetting nodes={
					if n=1
						{insert before={[,phantom]}}
						{}
				},
			fit=band,
			s sep=8pt,
			before computing xy={l=15pt},
			}
			[root, fill=gray!20
			[/frontend, for tree={fill=orange!20}
			[/src
			[/components]
			[/model]
			[/pages]
			[/services]
			[/styles]
			[/utils]
			]
			[Dockerfile]
			[package.json]
			[next.config.js]
			]
			[/datasets, for tree={fill=red!20}
				[/BLM20 - Reddit Collection]
				[/PAN13 - Author Profiling]
				[/PAN14 - Author Profiling]
				[/PAN15 - Author Profiling]
				[/PAN19 - Celebrity Profiling]
				[/PAN19 - Bots and Gender Profiling]
			]
			]
		\end{forest}
	\end{subfigure}
	\caption{Estructura de directorios del proyecto}
	\label{fig:estructura_proyecto}
\end{figure}

\section{Publicación del código fuente}
\label{sec:codigo_abierto}

{
	\color{blue}
	Una vez explicado el funcionamiento a bajo nivel de la aplicación, es necesario valorar la forma de publicar y
	distribuír el código fuente de la misma.

	\bigskip
	En este sentido, existen dos enfoques principales: el código abierto y el código privado.
	El código privado (\textit{closed source} o \textit{propietary source} en inglés) es aquel que no está disponible para el público en general, es decir, que no se puede acceder a él ni
	modificarlo, por lo que normalmente se asocia al mundo empresarial y a la venta de licencias para su uso.
	Por otro lado, el código abierto (\textit{open source} en inglés) es aquel que
	está disponible públicamente y puede ser modificado, utilizado y distribuído por cualquier persona en función
	de la licencia bajo la que esté publicado. Normalmente, el código abierto se asocia con la filosofía del \textit{software} libre
	y con la gratuidad de la aplicación, aunque esto no siempre es así.

	\bigskip
	En nuestro caso, creemos que la mejor forma de publicar el código fuente de la aplicación es hacerlo de forma abierta,
	favoreciendo a una mejora continua de la misma por parte de la comunidad y a ofrecer la posibilidad de que cualquiera pueda
	utilizarla y adaptarla a sus necesidades.

	\bigskip
	Sin embargo, para que esto sea posible, como se mencionó anteriormente, es necesario elegir una licencia bajo la cual publicarla. Así, existen licencias
	como la GPL (\textit{General Public License} en inglés) \cite{gpl} que permiten a los usuarios hacer uso del código, modificarlo y distribuirlo
	siempre y cuando se mantenga la misma licencia. Por otro lado, existen licencias como la MIT \cite{mitlicense} o la Apache \cite{apachelicense} que ofrecen una mayor libertad,
	permitiendo incluso la utilización del código en proyectos privados. Por lo tanto, teniendo en cuenta que la filosofía de este proyecto es la de ofrecer una herramienta útil y gratuita a la mayor
	cantidad de usuarios, sin importar si sus fines son comerciales o no, se ha optado por utilizar la licencia MIT.

	\bigskip
	En lo que respecta al lugar de publicación, se ha decidido hacer público el repositorio almacenado en GitHub \cite{github}, una de las
	plataformas más populares para el alojamiento de proyectos de software con más de 100 millones de desarrolladores \cite{100milliongithub}.
	Además, GitHub ofrece una gran cantidad de herramientas que facilitan el desarrollo colaborativo, como la posibilidad de crear \textit{issues} para
	reportar errores o sugerir mejoras o la opción realizar \textit{pull requests} para proponer cambios en el código.

	\bigskip
	Por úlitmo, para que la comunidad pueda contribuir al proyecto de la mejor forma posible, se ha creado una guía de contribución
	en la que se detallan los pasos a seguir para proponer cambios en el código, así como las normas de estilo que se deben cumplir.

	\bigskip
	Para acceder al repositorio, se puede hacer uso del siguiente enlace: \url{https://github.com/daavidrgz/ai-profiler}.
}
