\chapter{Planificación y costes}
\label{chap:planificacion}

\lettrine{E}{n} este capítulo se expondrá la planificación y distribución del trabajo en cada \textit{sprint}. Asimismo,
se detallará el coste asociado a todos los recursos involucrados en el proceso de desarrollo.

\section{Planificación temporal}
\label{sec:planificacion_temporal}

Con todo, el proyecto se ha dividido finalmente en seis \textit{sprints} de aproximadamente tres semanas de duración cada uno.
Se estima que las horas diarias de trabajo de un desarrollador sean 4 horas de media por lo que, teniendo en cuenta los fines de semana, se obtiene
un total de 60 horas por \textit{sprint} y 360 horas en total. En cuanto al \textit{Project Manager}, se estima que dedica 2 horas por \textit{sprint}
a la gestión del proyecto, lo que supone un total de 12 horas.

\begin{itemize}
	\item \textbf{Sprint 1}: Durante el primer sprint, del 15 de abril al 5 de mayo, se realizó la instalación del entorno
	      de desarrollo sobre el que se trabajaría posteriormente y se familiarizó con las librería más comunes de NLP.
	\item \textbf{Sprint 2}: Del 6 al 26 de mayo, se llevó a cabo una investigación de los posibles algoritmos a utilizar
	      junto a su rendimiento e implementación. A mayores, se buscó replicar los resultados que reportaban dichos algoritmos
	      en sus publicaciones y se seleccionaron los que mejor rendimiento mostraban. Este proceso se detalla más a fondo en las Secciones \ref{sec:competiciones_analizadas} y \ref{sec:algoritmos_seleccionados}.
	\item \textbf{Sprint 3}: Durante el tercer \textit{sprint}, del 27 de mayo al 16 de junio, se comenzó con el desarrollo
	      de la interfaz de usuario, es decir, diseño de \textit{mockups} e implementación de los componentes principales.
	\item \textbf{Sprint 4}: Del 17 de junio al 7 de julio, se continuó con el desarrollo de la interfaz de usuario y se integraron
	      en el \textit{backend} ambos algoritmos seleccionados.
	\item \textbf{Sprint 5}: Durante el quinto \textit{sprint}, del 8 al 28 de julio, se finalizó el desarrollo de la interfaz web
	      y el desarrollo se enfocó en estructurar el código del \textit{backend} así como de agregar la integración con la base de datos.
	      Además, se comenzó la redacción de esta memoria.
	\item \textbf{Sprint 6}: Finalmente, todo el mes de agosto se centró en la elaboración de la memoria y en pulir los últimos detalles
	      para la entrega de la aplicación.
\end{itemize}

\section{Recursos y costes}
\label{sec:planificacion_costes}

A lo largo del proyecto, se han empleado recuros de tres tipos diferentes: humanos, materiales y software.
Como recursos de tipo humano,
contamos con un grupo de trabajo de tres personas, el autor de este documento y los tutores del trabajo,
como se mecionó anteriormente en la Sección \ref{sec:metodologia_roles}.
En cuanto a los recursos de tipo software, todos los programas, librerías y herramientas en general
se han descrito en detalle en el Capítulo \ref{chap:herramientas}.
Finalmente, en lo que respecta a los recursos materiales asociados al proyecto, se ha hecho uso del portátil personal del autor,
cuyas especificaciones se muestran en la Tabla \ref{tab:costes_hardware}.

\bigskip
\begin{table}[H]
	\centering
	\rowcolors{2}{white}{udcgray!25}
	\begin{tabular}{|l|l|}
		\rowcolor{udcpink!25}
		\hline
		\small \textbf{Componente}        & \small \textbf{Modelo}                      \\ \hline
		\small \textit{Procesador}        & \small Intel Core™ i5-9300H @ 2.40GHz 4C/8T \\ \hline
		\small \textit{Memoria RAM}       & \small 16GB 2667MHz DDR4                    \\ \hline
		\small \textit{Disco duro}        & \small 512GB SSD NVMe M.2                   \\ \hline
		\small \textit{Tarjeta gráfica}   & \small NVIDIA GeForce GTX 1660Ti            \\ \hline
		\small \textit{Sistema operativo} & \small Arch Linux 6.1.12                    \\ \hline
	\end{tabular}
	\caption{Especificaciones del portátil empleado para el desarrollo del proyecto}
	\label{tab:costes_hardware}
\end{table}

\bigskip
En lo que respecta a los costes, dado que todos los recursos de tipo software utilizados son gratuitos, no se ha incurrido en ningún coste
asociado a ellos.

\bigskip
Por otro lado, ya que se hizo uso de un recurso material,
es necesario calcular su amortización asociada. Para ello, según la Ley 27/2014 del Impuesto sobre Sociedades (LIS) \cite{leysociedades},
los equipos para procesos de información tienen asociado un porcentaje anual de amortización del 25\% sobre el coste inicial del bien,
por lo que, dado que el valor en el momento de compra del portátil fue de aproximadamente 1,000€, la amortización anual ascendería a 250€. Este número, sin embargo,
tiene que ser proporcional a su tiempo de uso, cercano a los cuatro meses, por lo que el resultado final sería de 83.30€.

\bigskip
Finalmente, para estimar el coste de los recursos
humanos utilizados, se ha tenido en cuenta el salario medio de empleados con el mismo puesto en España. Según publica
uno de los portales de empleo más populares a nivel mundial, Indeed, el salario medio de un ingeniero de software que desarrolla
utilizando Python es de alrededor de 32,100€ anuales \cite{indeedsalario}. Además, el salario medio de un \textit{Project Manager}
del sector de las TIC es de unos 43,500€ anuales. Con estos datos y junto al cómputo total de horas calculado en la Sección \ref{sec:planificacion_temporal},
el coste del proyecto se puede ver detallado en la Tabla \ref{tab:costes}.

\bigskip
\begin{table}[H]
	\centering
	\rowcolors{2}{white}{udcgray!25}
	\begin{tabular}{|l|l|l|l|l|}
		\rowcolor{udcpink!25}
		\hline
		\small \textbf{Recurso}         & \small \textbf{Coste por hora} & \small \textbf{Horas} & \small \textbf{Total}     \\ \hline
		\small \textit{Desarrollador}   & \small 17.8€/h                 & \small 360h           & \small 6,408€             \\ \hline
		\small \textit{Project Manager} & \small 2 x 24.17€/h            & \small 12h            & \small 580€               \\ \hline
		\small \textit{Software}        & \small -                       & \small -              & \small 0€                 \\ \hline
		\small \textit{Materiales}      & \small -                       & \small -              & \small 83.30€             \\ \hline
		\small \textit{Total}           & \small -                       & \small -              & \small \textbf{7,071.30€} \\ \hline
	\end{tabular}
	\caption{Coste del proyecto}
	\label{tab:costes}
\end{table}
